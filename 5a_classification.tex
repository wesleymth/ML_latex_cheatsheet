\section*{Classification}
- Linear Decision boundaries: Assume we restrict ourselves to linear decision boundaries (hyperplane):

$\Rightarrow$ Prediction: $f(x)=\operatorname{sign}\left(x^{\top} w\right)$

- Assume the data are linearly separable, i.e., a separating hyperplane exists


\subsection*{Margin}
- Key concept: The margin is the distance from the hyperplane to the closest point


$\Rightarrow$ Take the one with the largest margin!

- Max-margin separating hyperplane: 
Choose the hyperplane which maximizes the margin
Why: If we slightly change the training set, the number of misclassifications will stay low

$\Rightarrow$ It will lead us to support vector machine (SVM) and logistic regression


\subsection*{Formalizing Binary Classification}

- Setting: $(X, Y) \sim \mathscr{D}$ with ranges $\mathscr{X}, \mathscr{Y}=\{-1,1\}$

- Loss function: (0-1 Loss) $\ell\left(y, y^{\prime}\right)=1_{y \neq y^{\prime}}= \begin{cases}1 & \text { if } y \neq y^{\prime} \\ 0 & \text { if } y=y^{\prime}\end{cases}$

- True risk for the classification:

$
L_{\mathscr{D}}(f)=\mathbb{E}_{\mathscr{D}}\left[1_{Y \neq f(X)}\right]=\mathbb{P}_{\mathscr{D}}[Y \neq f(X)]
$ (resp. classification error = probability of making an error)

- Goal: minimize $L_{\mathscr{D}}(f)$

\subsection*{Bayes classifier}

- Def: The classifier $f_{*}=\arg \min L_{\mathscr{D}}(f)$ is called the Bayes classifier $f$

- Claim:
$
f_{*}(x)=\arg \max _{y \in\{-1,1\}} \mathbb{P}(Y=y \mid X=x)
$

- Note: Bayes classifier is an unattainable gold standard, as we never know the underlying data distribution $\mathscr{D}$ in practice

\subsection*{Proof of the Bayes classifier}
- Claim 1: $\forall x \in \mathscr{X}, f_{*}(x) \in \arg \min _{y \in \mathscr{Y}} \mathbb{P}(Y \neq y \mid X=x) \Longrightarrow f_{*} \in \arg \min _{f: \mathscr{X} \rightarrow \mathscr{Y}} L_{\mathscr{D}}(f)$

$
L_{\mathscr{D}}(f)=\mathbb{E}_{X, Y}\left[1_{Y \neq f(X)}\right] =\mathbb{E}_{X}\left[\mathbb{E}_{Y \mid X}\left[1_{Y \neq f(X)} \mid X\right]\right] \\
=\mathbb{E}_{X}[\mathbb{P}(Y \neq f(X) \mid X)] \\
\geq \mathbb{E}_{X}\left[\min _{y \in \mathscr{Y}} \mathbb{P}(Y \neq y \mid X)\right] \\
=\mathbb{E}_{X}\left[\mathbb{P}\left(Y \neq f_{*}(X) \mid X\right)\right]=\mathbb{E}_{X, Y}\left[1_{Y \neq f_{*}(X)}\right]=L_{\mathscr{D}}\left(f_{*}\right)
$

- Claim 2: $f_{*}(x)=\arg \min _{y \in \mathscr{Y}} \mathbb{P}(Y \neq y \mid X=x)$

$
f_{*}(x)=\arg \max _{y \in \mathscr{Y}} \mathbb{P}(Y=y \mid X=x)=\arg \min _{y \in \mathscr{Y}} \mathbb{P}(Y \neq y \mid X=x)
$

\subsection*{Two classes of classification algorithms}
- Non-parametric: approximate the conditional distribution $\mathbb{P}(Y=y \mid X=x)$ via local averaging $\Rightarrow$ KNN

- Parametric: approximate true distribution $\mathscr{D}$ via training data $\Rightarrow$ Minimize the empirical risk on training data (ERM)

\subsection*{Classification by empirical risk minimization}
- How: minimize the empirical risk instead of the true risk:

$
\min _{f: X \rightarrow Y} L_{\operatorname{train}}(f):=\frac{1}{N} \sum_{n=1}^{N} 1_{f\left(x_{n}\right) \neq y_{n}}=\frac{1}{N} \sum_{n=1}^{N} 1_{y_{n} f\left(x_{n}\right) \leq 0}
$

- Problem: $L_{\text {train }}$ is not convex:

The set of classifiers is not convex because $\mathcal{Y}$ is discrete

The indicator function 1 is not convex because it is not continuous

\subsection*{Convex relaxation of the classification risk}

- Instead of learning $f: \mathscr{X} \rightarrow \mathscr{Y}$, learn $g: \mathscr{X} \rightarrow \mathbb{R}$ in a convex subset of continuous functions $\mathscr{G}$, and predict with $f(x)=\operatorname{sign}(g(x))$. The problem becomes

$
\min _{g \in \mathscr{G}} \frac{1}{N} \sum_{n=1}^{N} 1_{y_{n} g\left(x_{n}\right) \leq 0}
$

- Replace the indicator function by a convex $\phi: \mathbb{R} \rightarrow \mathbb{R}$ and minimize

$
\min _{g \in \mathscr{G}} \frac{1}{N} \sum_{n=1}^{N} \phi\left(y_{n} g\left(x_{n}\right)\right)
$

$\phi$ is a function of the functional margin $y_{n} g\left(x_{n}\right)$

$\Rightarrow$ This is a convex problem!

\begin{wrapfigure}{r}{0.5\columnwidth} 
    \includegraphics*[width=0.5\columnwidth]{figures/classification_losses2.jpg}
\end{wrapfigure}

Remark: possible to bound the zero-one risk $L(f)$ by the $\phi$ risk


\subsection*{Losses for Classification}

- Logistic loss $\rightarrow$ logistic regression

- Hinge loss $\rightarrow$ max margin classification


\subsection*{Good regressor $\Rightarrow$ good classifier}
- Consider $\mathscr{Y}=\{0,1\}$, for all regression functions $\eta: \mathscr{X} \rightarrow \mathbb{R}$ we can define a classifier as

$
\begin{aligned}
\mathscr{X} & \rightarrow\{0,1\} \\
f_{\eta}: x & \mapsto 1_{\eta(x) \geq 1 / 2}
\end{aligned}
$

- Claim:

$
L_{\mathscr{D}}^{\text {classif }}\left(f_{\eta}\right)-L_{\mathscr{D}}^{\text {classif }}\left(f^{*}\right) \leq 2 \sqrt{L_{\mathscr{D}}^{\ell_{2}}(\eta)-L_{\mathscr{D}}^{\ell_{2}}\left(\eta^{*}\right)}
$

Where $L_{\mathscr{D}}^{\text {classif }}\left(f_{\eta}\right)=\mathbb{E}_{\mathscr{D}}\left[1_{f(X) \neq Y}\right], L_{\mathscr{D}}^{\ell_{2}}(f)=\mathbb{E}_{\mathscr{D}}\left[(Y-f(X))^{2}\right]$ and $\eta_{*}=\arg \min _{\eta} L_{\mathscr{D}}^{\ell_{2}}(\eta)$

$\Rightarrow$ If $\eta$ is good for regression then $f_{\eta}$ is good for classification too (converse is not true)

