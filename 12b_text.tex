\section*{Text Representation}

-Finding numerical representations for words is fundamental for all machine learning methods dealing with text data.

-Goal: For each word, find mapping (embedding) $w_{i} \mapsto \mathbf{w}_{i} \in \mathbb{R}^{K}$

\subsection*{Co-Occurence Matrix}

-A big corpus of un-labeled text can be represented as the co-occurrence counts. $n_{i j}:=$ \#contexts where word $w_{i}$ occurs together with word $w_{j}$.

- Needs definition of Context e.g. document, paragraph, sentence, window and Vocabulary $\mathcal{V}:=\left\{w_{1}, \ldots, w_{D}\right\}$

- For words $w_{d}=1,2, \ldots, D$ and context words $w_{n}=1,2, \ldots, N$, the co-occurence counts $n_{i j}$ form a very sparse $D \times N$ matrix.

\subsection*{Learning Word-Representations}

- Find a factorization of the cooccurence matrix!

- Typically uses $\log$ of the actual counts, i.e. $x_{d n}:=\log \left(n_{d n}\right)$.

- Aim to find $\mathbf{W}, \mathbf{Z}$ s.t. $
\mathbf{X} \approx \mathbf{W} \mathbf{Z}^{\top}
$

$
\min _{\mathbf{W}, \mathbf{Z}} \mathcal{L}(\mathbf{W}, \mathbf{Z}):=\\\frac{1}{2} \sum_{(d, n) \in \Omega} f_{d n}\left[x_{d n}-\left(\mathbf{W} \mathbf{Z}^{\top}\right) d n\right]^{2}
$

where $\mathbf{W} \in \mathbb{R}^{D \times K}$, $\mathbf{Z} \in \mathbb{R}^{N \times K}$, $K \ll$ $D, N$, $\Omega \subseteq[D] \times[N]$ indices of non-zeros of the count matrix $\mathbf{X}$, $f_{d n}$ are weights to each entry.

\subsection*{GloVe}
A variant of word2vec.

$f_{d n}:=\min \left\{1,\left(n_{d n} / n_{\max }\right)^{\alpha}\right\}, \quad \alpha \in[0 ; 1] \quad$ (e.g. $\alpha=\frac{3}{4}$)

\textbf{Note:} Choosing K; $K$ e.g. 50, 100, 500

\subsection*{Training}
- Stochastic Gradient Descent (SGD) ($\Theta(K)$ per step $\rightarrow$ easily parralelizable)

- Alternating Least-Squares (ALS)

\section*{Skip-Gram Model}
- Uses binary classification (logistic regression objective), to separate real word pairs $\left(w_{d}, w_{n}\right)$ from fake word pairs. Same inner product score $=$ matrix factorization.

- Given $w_{d}$, a context word $w_{n}$ is:

real $=$ appearing together in a context window of size 5

fake $=$ any word $w_{n^{\prime}}$ sampled randomly: Negative sampling (also: Noise Contrastive Estimation)

\section*{Learning Representations of Sentences \& Documents}
- Supervised: For a supervised task (e.g. predicting the emotion of a tweet), we can use matrix factorization or CNNs.
- Unsupervised: 

Adding or averaging (fixed, given) word vectors, 

Training word vectors such that adding/averaging works well

Direct unsupervised training for sentences (appearing together with context sentences) instead of words

\subsection*{Fast Text}
Matrix factorization to learn document/sentence representations (supervised).

Given a sentence $s_{n}=$ $\left(w_{1}, w_{2}, \ldots, w_{m}\right)$, let $\mathbf{x}_{n} \in \mathbb{R}^{|\mathcal{V}|}$ be the bag-of-words representation of the sentence.

$$
\min _{\mathbf{W}, \mathbf{Z}} \mathcal{L}(\mathbf{W}, \mathbf{Z}):=\sum_{s_{n} \text { a sentence }} f\left(y_{n} \mathbf{W} \mathbf{Z}^{\top} \mathbf{x}_{n}\right)
$$

where $\mathbf{W} \in \mathbb{R}^{1 \times K}, \mathbf{Z} \in \mathbb{R}^{|\mathcal{V}| \times K}$ are the variables, and the vector $\mathbf{x}_{n} \in \mathbb{R}^{|\mathcal{V}|}$ represents our $n$-th training sentence.

Here $f$ is a linear classifier loss function, and $y_{n} \in\{ \pm 1\}$ is the classification label for sentence $\mathbf{x}_{n}$.


\subsection*{Language Models}
\subsubsection*{Selfsupervised training:}
- Can a model generate text? train classifier to predict the continuation (next word) of given text

- Multi-class:
Use soft-max loss function with a large number of classes $D=$ vocabulary size

- Binary classification: 
Predict if next word is real or fake (i.e. as in word2vec)

- Impressive recent progress using large models, such as transformers
