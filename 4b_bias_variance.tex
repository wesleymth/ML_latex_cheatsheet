\section*{Bias Variance Decomposition}
- Simple models have large bias but low variance

- Complex models have low bias but high variance

- We need to balance bias \& variance correctly

- Data model: output perturbed by some noise

$y=f(x)+\varepsilon $

$\varepsilon\sim\mathscr{D}_\varepsilon \approx$ i.i.d. independent of x $\mathbb{E}[\varepsilon]=0$

- We consider the square loss and will provide a decomposition of the true error

\subsection*{Error Decomposition}

- We are interested in how the expected error of $f_{S}$:
$
\mathbb{E}_{(x, y) \sim D}\left[\left(y-f_{S}(x)\right)^{2}\right]
$

- The decomposition will hold true at every single point $x$. Therefore, to simplify, we consider the expected error of $f_{S}$ for a fixed element $x_{0}$ :

$
L\left(f_{S}\right)=\mathbb{E}_{\varepsilon \sim \mathscr{D}_{\varepsilon}}\left[\left(f\left(x_{0}\right)+\varepsilon-f_{S}\left(x_{0}\right)\right)^{2}\right]
$

- This is a random variable. The randomness comes for the train set $S$

- We run the experiment many times, and we are interested in the average and the variance of the predictions $\left(f_{S_{1}}, \cdots, f_{S_{k}}\right)$ over these multiple runs

\subsection*{A decomposition in three terms}
We are interested in the expectation of the true risk over the training set $S$

$\mathbb{E}_{S \sim \mathscr{D}}[L(f_{S})] \\= \mathbb{E}_{S \sim \mathscr{D}}[\mathbb{E}_{\varepsilon \sim D_{\varepsilon}}[(f(x_{0})+\varepsilon-f_{S}(x_{0}))^{2}]] \\= \mathbb{E}_{S \sim \mathscr{D}, \varepsilon \sim \mathscr{D}_{\varepsilon}}[(f(x_{0})+\varepsilon-f_{S}(x_{0}))^{2}]$


$
\begin{aligned}
& =\mathbb{E}_{\varepsilon \sim \mathscr{D}_{\varepsilon}}\left[\varepsilon^{2}\right] \\
&+2 \mathbb{E}_{S \sim \mathscr{D}, \varepsilon \sim \mathscr{D}_{\varepsilon}}\left[\varepsilon\left(f\left(x_{0}\right)-f_{S}\left(x_{0}\right)\right)\right] \\
&+\mathbb{E}_{S \sim \mathscr{D}}\left[\left(f\left(x_{0}\right)-f_{S}\left(x_{0}\right)\right)^{2}\right]
\end{aligned}
$

Using that $\mathbb{E}_{\varepsilon \sim \mathscr{D}}[\varepsilon]=0$ and $\varepsilon \perp\perp S$ :

\begin{itemize}
  \item $\mathbb{E}_{\varepsilon \sim \mathscr{D}_{\varepsilon}}\left[\varepsilon^{2}\right]=\operatorname{Var}_{\varepsilon \sim \mathscr{D}_{\varepsilon}}[\varepsilon]$
  \item $\mathbb{E}_{S \sim \mathscr{D}, \varepsilon \sim \mathscr{D}_{\varepsilon}}\left[\varepsilon\left(f\left(x_{0}\right)-f_{S}\left(x_{0}\right)\right)\right]=\mathbb{E}_{\varepsilon \sim \mathscr{D} \varepsilon}[\varepsilon] \times \mathbb{E}_{S \sim \mathscr{D}}\left[f\left(x_{0}\right)-f_{S}\left(x_{0}\right)\right]=0$
\end{itemize}

$\Longrightarrow$

$\mathbb{E}_{S \sim \mathscr{D}, \varepsilon \sim \mathscr{D}_{\varepsilon}}\left[\left(f\left(x_{0}\right)+\varepsilon-f_{S}\left(x_{0}\right)\right)^{2}\right]=\operatorname{Var}_{\varepsilon \sim D_{\varepsilon}}[\varepsilon]+\mathbb{E}_{S \sim \mathscr{D}}\left[\left(f\left(x_{0}\right)-f_{S}\left(x_{0}\right)\right)^{2}\right]$

Trick: we add and subtract the constant term $\mathbb{E}_{S^{\prime} \sim D}\left[f_{S^{\prime}}\left(x_{0}\right)\right]$, where $S^{\prime}$ is a second training set independent from $S$

\scalebox{0.50}{
    $
\begin{aligned}
\mathbb{E}_{S \sim \mathscr{D}}\left[\left(f\left(x_{0}\right)-f_{S}\left(x_{0}\right)\right)^{2}\right]= & \mathbb{E}_{S \sim \mathscr{D}}\left[\left(f\left(x_{0}\right)-\mathbb{E}_{S^{\prime} \sim \mathscr{D}}\left[f_{S^{\prime}}\left(x_{0}\right)\right]+\mathbb{E}_{S^{\prime} \sim \mathscr{D}}\left[f_{S^{\prime}}\left(x_{0}\right)\right]-f_{S}\left(x_{0}\right)\right)^{2}\right] \\
= & \mathbb{E}_{S \sim \mathscr{D}}\left[\left(f\left(x_{0}\right)-\mathbb{E}_{S^{\prime} \sim \mathcal{D}}\left[f_{S^{\prime}}\left(x_{0}\right)\right]\right)^{2}+\left(\mathbb{E}_{S^{\prime} \sim \mathscr{D}}\left[f_{S^{\prime}}\left(x_{0}\right)\right]-f_{S}\left(x_{0}\right)\right)^{2}\right. \\
& \left.+2\left(f\left(x_{0}\right)-\mathbb{E}_{S^{\prime} \sim \mathscr{D}}\left[f_{S^{\prime}}\left(x_{0}\right)\right]\right)\left(\mathbb{E}_{S^{\prime} \sim \mathscr{D}}\left[f_{S^{\prime}}\left(x_{0}\right)\right]-f_{S}\left(x_{0}\right)\right)\right]
\end{aligned}
$
}


Cross-term:

\scalebox{0.6}{
$
\begin{aligned}
& \mathbb{E}_{S \sim \mathscr{D}} {\left[\left(f\left(x_{0}\right)-\mathbb{E}_{S^{\prime} \sim \mathscr{D}}\left[f_{S^{\prime}}\left(x_{0}\right)\right]\right) \cdot\left(\mathbb{E}_{S^{\prime} \sim \mathscr{D}}\left[f_{S^{\prime}}\left(x_{0}\right)\right]-f_{S}\left(x_{0}\right)\right)\right] } \\
& \quad=\left(f\left(x_{0}\right)-\mathbb{E}_{S^{\prime} \sim \mathscr{D}}\left[f_{S^{\prime}}\left(x_{0}\right)\right]\right) \cdot \mathbb{E}_{S \sim \mathscr{D}}\left[\left(\mathbb{E}_{S^{\prime} \sim \mathscr{D}}\left[f_{S^{\prime}}\left(x_{0}\right)\right]-f_{S}\left(x_{0}\right)\right)\right] \\
& \quad=\left(f\left(x_{0}\right)-\mathbb{E}_{S^{\prime} \sim \mathscr{D}}\left[f_{S^{\prime}}\left(x_{0}\right)\right]\right) \cdot\left(\mathbb{E}_{S^{\prime} \sim \mathcal{D}}\left[f_{S^{\prime}}\left(x_{0}\right)\right]-\mathbb{E}_{S \sim \mathscr{D}}\left[f_{S}\left(x_{0}\right)\right]\right)=0 .
\end{aligned}
$
}

$
\Longrightarrow
\mathbb{E}_{S \sim \mathscr{D}}\left[\left(f\left(x_{0}\right)-f_{S}\left(x_{0}\right)\right)^{2}\right]
\\=\left(f\left(x_{0}\right)-\mathbb{E}_{S^{\prime} \sim \mathscr{D}}\left[f_{S^{\prime}}\left(x_{0}\right)\right]\right)^{2}
\\ \qquad + \mathbb{E}_{S \sim \mathscr{D}}\left[\left(\mathbb{E}_{S^{\prime} \sim \mathscr{D}}\left[f_{S^{\prime}}\left(x_{0}\right)\right]-f_{S}\left(x_{0}\right)\right)^{2}\right]
$

\subsection*{Bias-Variance Decomposition}
- We obtain the following decomposition into three positive terms:

$\mathbb{E}_{S \sim \mathscr{D}, \varepsilon \sim \mathscr{D}_{\varepsilon}}\left[\left(f\left(x_{0}\right)+\varepsilon-f_{S}\left(x_{0}\right)\right)^{2}\right]
\\=\operatorname{Var}_{\varepsilon \sim \mathscr{D}_{\varepsilon}}[\varepsilon] \leftarrow \text { Noise variance }$ 

$
\begin{aligned}
&+\left(f\left(x_{0}\right)-\mathbb{E}_{S^{\prime} \sim \mathscr{D}}\left[f_{S^{\prime}}\left(x_{0}\right)\right]\right)^{2} \leftarrow \text { Bias } 
\\
& +\mathbb{E}_{S \sim \mathscr{D}}\left[\left(f_{S}\left(x_{0}\right)-\mathbb{E}_{S^{\prime} \sim \mathscr{D}}\left[f_{S^{\prime}}\left(x_{0}\right)\right]\right)^{2}\right] \leftarrow \text { Var. } 
\end{aligned}
$

- each of which always provides a lower bound of the true error

- $\Rightarrow$ To minimize the true error, we must choose a method that achieves low bias and low variance simultaneously

\subsection*{Noise: $\operatorname{Var}_{\varepsilon \sim \mathscr{D}_{\varepsilon}}[\varepsilon]$}

- a strict lower bound on the achievable error

\begin{itemize}
  \item It is not possible to go below the noise level

  \item Even if we know the true model $f$, we still suffer from the noise: $L(f)=\mathbb{E}\left[\varepsilon^{2}\right]$

  \item It is not possible to predict the noise from the data since they are independent

\end{itemize}

\subsection*{Bias: $\left(f\left(x_{0}\right)-\mathbb{E}_{S \sim \mathscr{D}}\left[f_{S}\left(x_{0}\right)\right]\right)^{2}$}

\begin{itemize}
  \item Squared of the difference between the actual value $f\left(x_{0}\right)$ and the expected prediction

  \item It measures how far off in general the models' predictions are from the correct value

  \item If model complexity is low, bias is typically high

  \item If model complexity is high, bias is typically low

\end{itemize}

\subsection*{Variance: $\mathbb{E}_{S \sim \mathscr{D}}\left[\left(f_{S}\left(x_{0}\right)-\mathbb{E}_{S \sim D}\left[f_{S}\left(x_{0}\right)\right]\right)^{2}\right]$}

\begin{itemize}
  \item Variance of the prediction function
  \item It measures the variability of predictions at a given point across different training set realizations
  \item If we consider complex models, small variations in the training set can lead to significant changes in the predictions
\end{itemize}

\subsection*{Bias Variance tradeoff}

\includegraphics*[width=0.7\columnwidth]{figures/bias_variance.jpg}

\subsection*{Double descent curve}

\includegraphics*[width=0.7\columnwidth]{figures/double_descent_curve.jpg}
