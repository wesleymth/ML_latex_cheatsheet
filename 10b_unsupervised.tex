\section{Unsupervised learning}

- In unsupervised learning, our data consists only of features (or inputs) $\mathbf{x}_{1}, \mathbf{x}_{2}, \ldots, \mathbf{x}_{N}$, vectors in $\mathbb{R}^{D}$, and there are no outputs $y_{n}$ available.

- Unsupervised learning seems to play an important role in how living beings learn. Variants of it seem to be more common in the brain than supervised learning.

- Two main directions in unsupervised learning are

\begin{itemize}
  \item representation learning
  \item density estimation \& generative models
\end{itemize}

\subsection*{Examples for Representation Learning}
- Given ratings of movies and viewers, we use matrix factorization to extract useful features (see e.g. Netflix Prize). Maybe not unsupervised?

- Learning word-representations using matrix-factorizations, word2vec (Mikolov et al. 2013).

- Given voting patterns of regions across Switzerland, we use PCA to extract useful features (Etter et al. 2014).

- PCA Example 2: Genes mirror geography

- Dendrogram from agglomerative hierarchical clustering with average linkage to the human tumor microarray data.

- Clustering more than two million biomedical publications (Kevin Boyack et.al. 2011)

\subsection*{Unsupervised Representation Learning \& Generation}
\subsection*{How does it work?}
Define a unsupervised or self-supervised loss function, for

\begin{itemize}
  \item Compression \& Reconstruction (e.g. Auto-Encoder)

  \item Consistency \& Contrastive Learning (e.g. Noise-contrastive estimation)

  \item Generation (e.g. Auto-Encoder, Gaussian Mixture Model)
\end{itemize}

\subsection*{Examples:}
$(\mathrm{G}=$ can be used as a generative model $)$

\begin{itemize}
  \item Auto-Encoders (G): Invertible networks, learned compression, normalizing flows
  \item Representation Learning: e.g. images, text, time-series, video. Combining unsupervised representation learning (pre-training) with supervised learning

  \item Language Models \& Sequence Models (G): text generation, or sequence continuation, BERT, video, audio \& timeseries (auto-regressive, contrastive, ...)

  \item Generative Adversarial Networks (GAN) (G) see also predictability minimization
  
  \item Contrastive image-language pretraining (CLIP): learns a joint representation space for images and text using contrastive learning

  \item Diffusion models (G) new state-of-the-art in image generation; these models can be adapted to generate images from text prompts (e.g., DALL-E 2, Stable Diffusion, Midjourney)

\end{itemize}