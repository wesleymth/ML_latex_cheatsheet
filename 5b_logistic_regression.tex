
\section{Logistic Regression}
% - Binary classification : We observe some data $S=\left\{x_{n}, y_{n}\right\}_{n=1}^{N} \in \mathscr{X} \times\{0,1\}$


% \subsection*{Motivation for logistic regression}
% Instead of directly modeling the output $Y$, we can model the probability that $Y$ belongs to a specific class. Map the prediction from $(-\infty,+\infty)$ to $[0,1]$

% \subsection*{The logistic function}
Logistic function:
$
\sigma(\eta):=\frac{e^{\eta}}{1+e^{\eta}}=\frac{1}{1+e^{-\eta}}
$, 
$1-\sigma(\eta)=\left(1+e^{\eta}\right)^{-1}$, 
$\sigma^{\prime}(\eta)=\sigma(\eta)(1-\sigma(\eta))$

\subsection*{Logistic Regression}
$
p(1 \mid x):=\mathbb{P}(Y=1 \mid X=x)=\sigma\left(x^{\top} w+w_{0}\right) \\
p(0 \mid x):=\mathbb{P}(Y=0 \mid X=x)=1-\sigma\left(x^{\top} w+w_{0}\right)
$

% Logistic regression models the probability that $Y$ belongs to a particular class using the logistic function $\sigma$

% Label prediction: quantize the probability:

$
\text {If } p(1 \mid x) \geq 1 / 2 \text {, you predict the class } 1 \text { , else class } 0
$

% - Interpretation: large $\left|x^{\top} w+w_{0}\right|$ corresponds to $p(1 \mid x)$ very close to 0 or 1 (high confidence). Small $\left|x^{\top} w+w_{0}\right|$ corresponds to $p(1 \mid x)$ very close to .5 (low confidence)


% \subsection*{Comparison of logistic and linear regression for data with extreme values}

% - More robust to unbalanced data and extremes.

% \subsection*{Geometric Interpretation}

% - The vector $w$ is orthogonal to the "surface of transition"

% - The transition between the two levels happens at the hyperplane $w^{\perp}=\left\{v, v^{\top} w=0\right\}$

% - Scaling $w$ makes the transition faster or slower


% - Changing $w_{0}$ shifts the decision region along the $w$ vector

% - The transition happens at the hyperplane $\left\{v, v^{\top} w+w_{0}=0\right\}$

% - Bias term: should consider a shift $w_{0}$ as there is no reason for the transition hyperplane to pass through the origin:
% $
% p(1 \mid x)=\sigma\left(w^{\top} x+w_{0}\right)
% $
% For simplicity, add the constant 1 to the feature vector
% $
% x=(x, 1)^T
% $
% It is crucial for allowing to shift the decision region

\subsection*{MLE for logistic regression}
- Assumption: The inputs $\mathbf{X}$ do not depend on the parameter $w$ we choose:

$
\mathscr{L}(w)=p(\mathbf{y}, \mathbf{X} \mid w)=p(\mathbf{X} \mid w) p(\mathbf{y} \mid \mathbf{X}, w)
\\
\underset{\mathbf{X} \perp \perp w}{=} p(\mathbf{X}) p(\mathbf{y} \mid \mathbf{X}, w)p(\mathbf{y} \mid \mathbf{X}, w) 
=\Pi_{n=1}^{N} p\left(y_{n} \mid x_{n}, w\right) 
=\Pi_{n: y_{n}=1} p\left(y_{n}=1 \mid x_{n}, w\right) 
\qquad \times \Pi_{n: y_{n}=0} p\left(y_{n}=0 \mid x_{n}, w\right) 
=\Pi_{n=1}^{N} \sigma\left(x_{n}^{\top} w\right)^{y_{n}}\left[1-\sigma\left(x_{n}^{\top} w\right)\right]^{1-y_{n}}
$

$
\mathscr{L}(w) \propto \prod_{n=1}^{N} \sigma\left(x_{n}^{\top} w\right)^{y_{n}}\left[1-\sigma\left(x_{n}^{\top} w\right)\right]^{1-y_{n}}
$

\subsection*{Minimum of the Negative Log Likelihood (NLL)}
$
-\log (p(\mathbf{y} \mid \mathbf{X}, w))
=-\log \left(\prod_{n=1}^{N} \sigma\left(x_{n}^{\top} w\right)^{y_{n}}\left[1-\sigma\left(x_{n}^{\top} w\right)\right]^{1-y_{n}}\right) \\
% =-\sum_{n=1}^{N} y_{n} \log \sigma\left(x_{n}^{\top} w\right)+\left(1-y_{n}\right) \log \left(1-\sigma\left(x_{n}^{\top} w\right)\right) \\
% =\sum_{n=1}^{N} y_{n} \log \left(\frac{1-\sigma\left(x_{n}^{\top} w\right)}{\sigma\left(x_{n}^{\top} w\right)}\right)-\log \left(1-\sigma\left(x_{n}^{\top} w\right)\right) \\
=\sum_{n=1}^{N}-y_{n} x_{n}^{\top} w+\log \left(1+e^{x_{n}^{\top} w}\right) 
% \leftarrow 1-\sigma(\eta)=\frac{1}{1+e^{\eta}} \Rightarrow \frac{1-\sigma(\eta)}{\sigma(\eta)}=e^{-\eta}
$

$
w_{*}=\arg \min L(w) :=\frac{1}{N} \sum_{n=1}^{N}-y_{n} x_{n}^{\top} w+\log \left(1+e^{x_{n}^{\top} w}\right)
$

% \subsection*{A side note on logistic loss}
- NLL is equivalent to ERM for the logistic loss 
% (a surrogate for $0-1$ loss, as discussed yesterday)

- $y \in\{0,1\}$ :
$
\ell(y, g(x))=-y g(x)+\log (1+\exp (g(x)))
$

- $y \in\{-1,1\}$ :
$
\ell(y, g(x))=\log (1+\exp (-y g(x)))
$

% - Note: the logistic loss can be applied in modern machine learning as well: $g(x)$ can represent the output of a neural network

\subsection*{Gradient of the negative log likelihood}

$\nabla L(w)=\nabla\left[\frac{1}{N} \sum_{n=1}^{N} \log \left(1+e^{x_{n}^{\top} w}\right)-y_{n} x_{n}^{\top} w\right]=\frac{1}{N} \sum_{n=1}^{N} \frac{e^{x_{n}^{\top} w} x_{n}}{1+e^{x_{n}^{\top} w}}-y_{n} x_{n}=\frac{1}{N} \sum_{n=1}^{N}\left(\sigma\left(x_{n}^{\top} w\right)-y_{n}\right) x_{n}$

$
\nabla L(w)=\frac{1}{N} \mathbf{X}^{\top}(\sigma(\mathbf{X} w)-\mathbf{y})
$

% - Same gradient as in LS but with $\sigma$

% - No closed form solution to $\nabla L(w)=0$ but $L$ is convex

\subsection*{Convexity of the loss function $L$}
% $
% L(w)=\frac{1}{N} \sum_{n=1}^{N}-y_{n} x_{n}^{\top} w+\log \left(1+e^{x_{n}^{\top} w}\right)
% $
% is convex w.r.t. $w$

% % - Proof: $L$ is obtained through simple convexity preserving operations:

% 1) Positive combinations of convex functions is convex

% 2) Composition of a convex and a linear functions is convex

% 3) A linear function is both convex and concave

% 4) $h(\eta):=\log \left(1+e^{\eta}\right)$ is convex:

% $h^{\prime}(\eta)=\frac{e^{\eta}}{1+e^{\eta}}=\sigma(\eta)$, 
% $
% h^{\prime \prime}(\eta)=\sigma^{\prime}(\eta)=\frac{e^{\eta}}{\left(1+e^{\eta}\right)^{2}} \geq 0
% $

% 2) + 4) $ \Rightarrow
% \log \left(1+e^{x_{n}^{\top} w}\right) \text { is convex }
% $, 
% 3) $ \Rightarrow
% -y_{n} x_{n}^{\top} w \text { is convex }
% $, 
% 1) $ \Rightarrow
% L(w) \text{ is convex }
% $

% \subsection*{Second proof: Hessian of $L$ is psd}
- The Hessian $\nabla^{2} L$, $\frac{\partial^{2}}{\partial w_{i} \partial w_{j}} L(w)$, is psd $\Rightarrow$ convex

$
\nabla^{2} L(w) 
% =\nabla[\nabla L(w)]^{\top} =\nabla\left[\frac{1}{N} \sum_{n=1}^{N} x_{n}\left(\sigma\left(x_{n}^{\top} w\right)-y_{n}\right)\right]^{\top} =\frac{1}{N} \sum_{n=1}^{N} \nabla \sigma\left(x_{n}^{\top} w\right) x_{n}^{\top} 
=\frac{1}{N} \sum_{n=1}^{N} \sigma\left(x_{n}^{\top} w\right)\left(1-\sigma\left(x_{n}^{\top} w\right)\right) x_{n} x_{n}^{\top}
$

$
\nabla^{2} L(w)=\frac{1}{N} \mathbf{X}^{\top} S \mathbf{X} \text { where } S=\operatorname{diag}\left[\sigma\left(x_{n}^{\top} w\right)\left(1-\sigma\left(x_{n}^{\top} w\right)\right)\right] \succcurlyeq 0
$

$\Rightarrow \mathrm{L}$ is convex since $\nabla^{2} L(w) \geqslant 0$

% \subsection*{How to minimize the convex function L?}
% \underline{GD:}
% $
% \left\{\begin{array}{l}
% w_{0} \in \mathbb{R}^{d} \\
% w_{t+1}=w_{t}-\frac{\gamma_{t}}{N} \sum_{n=1}^{N}\left(\sigma\left(x_{n}^{\top} w_{t}\right)-y_{n}\right) x_{n}
% \end{array}\right.
% $,
% can be slow $\mathcal{O}(N\times T)$

% \underline{SGD:}
% $
% \left\{\begin{array}{l}
% w_{0} \in \mathbb{R}^{d} \\
% w_{t+1}=w_{t}-\gamma_{t}\left(\sigma\left(x_{n_{t}}^{\top} w_{t}\right)-y_{n_{t}}\right) x_{n_{t}}
% \end{array}\right. 
% % \text { where } \mathbb{P}\left[n_{t}=n\right]=1 / N
% $, 
% is faster but converges slower

\subsection*{Newton's method uses second order information}
- Newton's method minimizes the quadratic approximation:

$
L(w) \sim L\left(w_{t}\right)+\nabla L\left(w_{t}\right)^{\top}\left(w-w_{t}\right)+\frac{1}{2}\left(w-w_{t}\right)^{\top} \nabla^{2} L\left(w_{t}\right)\left(w-w_{t}\right) :=\phi_{t}(w) \\
\tilde{w}=\arg \min \phi_{t}(w) \Rightarrow \nabla L\left(w_{t}\right)+\nabla^{2} L\left(w_{t}\right)\left(\tilde{w}-w_{t}\right)=0
$

- Newton's method: $w_{t+1}=w_{t}-\gamma_{t} \nabla^{2} L\left(w_{t}\right)^{-1} \nabla L\left(w_{t}\right)$

- Step-size needed to ensure convergence (damped Newton's method)

- Convergence faster than GD but comp. complex. higher.

\subsection*{Problem when the data are linearly separable}

$
\inf _{w} L(w)=0=\lim _{\alpha \rightarrow \infty} L(\alpha \cdot \bar{w})
$ $\Rightarrow$ the weights will go to $\infty$

- Solution: add a $\ell_{2}$-regularization (Ridge logistic regression)
% - \underline{Ridge logistic regression}:
$\frac{1}{N} \sum_{n=1}^{N}-y_{n} x_{n}^{\top} w+\log \left(1+e^{x_{n}^{\top} w}\right)+\frac{\lambda}{2}\|w\|_{2}^{2}$

- Optimization perspective: stabilize the optimization process

- Statistical perspective: avoid overfitting
